\documentclass[a4paper,11pt]{article}
\usepackage[utf8x]{inputenc}
\usepackage{fancyhdr}
\usepackage[spanish]{babel}
\usepackage{lastpage}
\usepackage{amstext}
\usepackage{amsmath}
\usepackage{amsfonts}
\usepackage{amsthm}
\usepackage{amssymb}
\usepackage{enumerate}
\usepackage{graphicx}
\usepackage{etoolbox}
\usepackage{wasysym}
\usepackage{fancyvrb}
\usepackage[implicit=false]{hyperref}
\usepackage[a4paper, total={6.5in, 9.5in}]{geometry}
\usepackage[T1]{fontenc}
\usepackage[sc]{mathpazo}

\newcommand{\at}{@}

\title{Teoría de Números\\
      \small{Práctica 1}}
\author{Lucio Santi\\
        \texttt{lsanti\at dc.uba.ar}}
\date{\today}

\pagestyle{fancyplain} 
\renewcommand{\headrulewidth}{0pt}
\cfoot{\thepage/\pageref{LastPage}}
\lhead{}
\chead{}
\rhead{}

\newcommand{\Zm}[1]{\ensuremath{\mathbb{Z}[#1]}}
\newcommand{\Cong}[3]{\ensuremath{#1 \equiv #2 \, \textrm{mod } #3}}
\newcommand{\Div}[2]{\ensuremath{#1 | #2}}


\newtheorem*{ej}{Ejercicio}

\begin{document}
\maketitle

\begin{ej} 
    Asumiendo que el anillo \Zm{\frac{1 + \sqrt{-19}}{2}} es DFU, probar que $(x, y) = (\pm 18, 7)$
son las únicas soluciones enteras de la ecuación

    $$x^2 + 19 = y^3$$
\end{ej}

\begin{proof}[Resoluci\'on]
TBD
\end{proof}


\begin{ej} 
Caracterizaremos los primos que son suma de dos cuadrados. Probaremos que, si $p$ es un
primo impar, entonces $p = x^2 + y^2$ si y sólo si $\Cong{p}{1}{4}$.

\begin{enumerate}[i.]
    \item Sea $p$ primo impar tal que $p$ se escribe como suma de dos cuadrados.
    Probar que -1 es un cuadrado módulo $p$. Concluir que $\Cong{p}{1}{4}$.

    \item Sea $p$ primo impar, $\Cong{p}{1}{4}$. Tomar $n \in \mathbb{Z}$ tal que
    $\Cong{n^2}{-1}{p}$. Como $\Div{p}{n^2 + 1}$ en $\mathbb{Z}$, tenemos que
    $\Div{p}{(n + i)(n - i)}$ en \Zm{i}. Probar que $p$ no es primo de \Zm{i} y, 
    por lo tanto, es reducible.

    \item Sabiendo que $p = \alpha \dot \beta$, $\alpha, \beta \in \Zm{i}$ no 
    unidades, concluir que $p$ es suma de dos cuadrados.
\end{enumerate}

\end{ej}

\begin{proof}[Resoluci\'on]
TBD
\end{proof}


\begin{ej} 
Caracterización de los irreducibles de \Zm{i}.

\begin{enumerate}[i.]
    \item  Probar que $2 = (-i)(1+i)^2$ y que $1+i$ es irreducible.

    \item Sea \Cong{p}{3}{4}. Probar que $p = x^2 + y^2$ no tiene soluciones
    en $\mathbb{Z}$. Concluir que $p$ es irreducible en \Zm{i}.

    \item Utilizar el ejercicio anterior para probar que, si \Cong{p}{1}{4}, entonces
    $p$ se factoriza en \Zm{i} como producto de dos irreducibles no asociados.

    \item Probar que, si $\pi$ es un irreducible de \Zm{i}, entonces $\pi$ es asociado
    a alguno de los irreducibles mencionados en los ítems anteriores (sug.: si $\pi$ es
    irreducible, existe un primo $p \in \mathbb{Z}$ tal que \Div{p}{N(\pi) = \pi \bar{\pi}}
    y usar factorización única).
\end{enumerate}
\end{ej}

\begin{proof}[Resoluci\'on]
TBD
\end{proof}

\begin{ej} 
Factorizar como producto de irreducibles los elementos $7 + 4i$ y $23 + 14i$ en \Zm{i}.
\end{ej}

\begin{proof}[Resoluci\'on]
Inmediato usando SageMath \smiley:

\begin{center}
\begin{minipage}{6.8cm}
    \begin{Verbatim}[frame=single,fontsize=\footnotesize]
    sage: K.<i> = QuadraticField(-1)
    sage: factor(7 - 4*i)
    (i) * (-i - 2) * (2*i + 3)
    sage: factor(23+14*i)
    (-i - 2)^2 * (-2*i + 5)    
    \end{Verbatim}
\end{minipage}
  \end{center}

\end{proof}

\end{document}
